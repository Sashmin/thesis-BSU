\section{Заключение}

    Таким образом, в курсовой работе получены следующие основные результаты:
    \begin{enumerate}
        \item Произведено сравнение поведения оценок параметров распределения вероятности, полученных методом игнорирования пропусков, для различных видов пропусков во временном ряду. Показана несостоятельность метода в случае неслучайных пропусков. 
        \item На примере метода максимального правдоподобия исследовано влияние, оказываемое на стандартные методы получения оценок параметров смешением плотности распределения вероятности с функцией модели пропусков, зависящих от данных.
        \item Исследованы методы, позволяющие получить верные оценки параметров распределения для случая неслучайных пропусков.
    \end{enumerate}