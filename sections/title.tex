\begin{titlepage}
    \begin{center}
        \textbf{\textscale{0.9}{БЕЛОРУССКИЙ ГОСУДАРСТВЕННЫЙ УНИВЕРСИТЕТ\\[0.2cm]
        ФАКУЛЬТЕТ ПРИКЛАДНОЙ МАТЕМАТИКИ И ИНФОРМАТИКИ\\[0.2cm] }
        Кафедра математического моделирования и анализа данных
        }

        \vspace{4cm}
        \textbf{Отчет\\[0.4cm]
        о прохождении производственной (преддипломной) практики
        }
    \end{center}

    \vfill

    \begin{flushleft}
        \begin{adjustwidth}{7.2cm}{}
            Толочко Александра Викторовича\\[0.3cm]
            студента 4 курса,\\
            специальность Прикладная математика»\\[0.4cm]
            Руководитель практики:\\[0.3cm] 
            кандидат физ.-мат. наук,\\
            доцент И.А. Бодягин
        \end{adjustwidth}
    \end{flushleft}

    \vfill

    \begin{center}
        Минск, 2025
    \end{center}
\end{titlepage}

\begin{center}
    БЕЛОРУССКИЙ ГОСУДАРСТВЕННЫЙ УНИВЕРСИТЕТ\\[0.2cm]
    Факультет прикладной математики и информатики\\[0.2cm]
\end{center}
Кафедра $\underset{\text{(наименование кафедры)}}{\underline{\hspace{8cm}}}$

\vspace{1cm}
\hspace*{8cm}\textbf{Утверждаю}\\[0.2cm]
\hspace*{5cm}Заведующий кафедрой $\underset{\text{(подпись)(фамилия, инициалы)}}{\underline{\hspace{6cm}}}$\\[0.6cm]
\hspace*{10.8cm}<<$\underline{\hspace{0.5cm}}$>>$\underline{\hspace{3cm}}$20$\underline{\hspace{0.5cm}}$г.

\vspace{0.4cm}
\begin{center}
    \textbf{Задание на практику}\\[0.3cm]
    \textbf{по специальности <<Прикладная математика>>}
\end{center}
\begin{flushleft}
    {Студенту $\underset{\text{(фамилия, инициалы)}}{\underline{\hspace{6cm}}}$

    \vspace{0.6cm}
    1. Тема практики: $\underset{\text{(наименование темы дипломной работы)}}{\underline{\hspace{10cm}}}$\\

    \vspace{0.6cm}
    2. Список рекомендуемой литературы:\\
    \hspace*{0.8cm}2.1. Андерсон, Т. Статистический анализ временных рядов\\
    \hspace*{0.8cm}2.2. Дженкинс, Г., Ваттс, Д. Спектральный анализ и его приложения: в 2 т.\\
    \hspace*{0.8cm}2.3. Бокс Дж., Дженкинс Г. Анализ временных рядов, прогноз и управление\\
    \hspace*{0.8cm}2.4. Харин, Ю.С. Оптимальность и робастность в статистическом
    прогнозировании\\[0.3cm]

    3. Перечень подлежащих разработке вопросов или краткое содержание расчетно-пояснительной записки:\\[0.2cm]
    \hspace*{0.8cm}3.1. Исследовать применимость метода игнорирования пропусков в случае
    пропусков, зависящих от данных\\[0.2cm]
    \hspace*{0.8cm}3.2. Показать, как модель неслучайных пропусков влияет на работу стандартных
    методов оценивания параметров\\[0.2cm]
    \hspace*{0.8cm}3.3. Получить верные оценки параметров распределения в случае пропусков, 
    зависящих от данных\\[0.3cm]

    4. Примерный календарный график:\\[0.2cm]
    \begin{itemize}
        \item[$-$] \textbf{февраль (1-ая неделя)} $-$ ознакомление с условиями работы,
        изучение основных теоретических вопросов, получение задания.
        \item[$-$] \textbf{февраль (2-3-я неделя)} $-$ построение модели для описания различных 
        видов пропусков.
        \item[$-$] \textbf{март (4-5-ая неделя)} $-$ исследование метода максимального правдоподобия для получения оценки
        в случае пропусков, зависящих от данных.
        \item[$-$] \textbf{март (6-7-ая неделя)} $-$ исследование метода моментов для получения оценки
        в случае пропусков, зависящих от данных.
        \item[$-$] \textbf{апрель (8-9-ая неделя)} $-$ проведение комрьютерных экспериментов для
        проверки полученных результатов.
        \item[$-$] \textbf{апрель (10-ая неделя)} $-$ обобщение результатов и подготовка отчета
    \end{itemize}

    \vspace{0.4cm}
    5. Руководители практики:\\[0.2cm]
    \hspace*{0.2cm}от предприятия $\underset{\text{(ФИО)}}{\underline{\hspace{8cm}}}$\\[0.1cm]
    \hspace*{0.2cm}от кафедры $\underset{\text{(ФИО)}}{\underline{\hspace{8cm}}}$\\[0.3cm]

    \vspace{0.6cm}
    6. Дата выдачи задания \hspace{0.2cm} $\underline{\hspace{4cm}}$\\[0.2cm]

    7. Срок сдачи отчета \hspace{0.88cm} $\underline{\hspace{4cm}}$\\[0.6cm]
    
    {\small $\underset{\text{(от кафедры)}}{\text{Руководитель}} \underset{\text{(подпись)}}{\underline{\hspace{3cm}}}\;\underset{\text{(инициалы, фамилия)}}{\underline{\hspace{6cm}}}$\\[0.6cm]
    Подпись студента $\underset{\text{(подпись, дата)}}{\underline{\hspace{6cm}}}$}
    
    }
\end{flushleft}