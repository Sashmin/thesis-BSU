\section{Математическая модель и постановка задачи}

    Пусть наблюдается выборка
    \begin{equation*}
        X = (x_1, x_2, \dots , x_T), \quad x_i \in \Rb ,
    \end{equation*}
    из некоторого абсолютно непрерывного закона распределения вероятности с плотностью $p_\xi$, заданной с точностью 
    до параметров. И пусть некоторые значения выборки пропущены, то есть известно, что на 
    данном месте должно наблюдаться некоторое значение, но само оно неизвестно. В литературе чаще всего встречаются 
    два основных вида пропусков во временных рядах по зависимости от данных:
    \begin{enumerate}
        \item Случайные пропуски (не зависят от данных)
        \item Цензурирование (зависят от данных)
    \end{enumerate}

    Обобщим следующим образом. Пусть нам дана выборка вида
    \begin{equation*}
        X = (x_1, x_2, \dots , x_T), \quad x_i \in \Rb \cup \{ NA \},
    \end{equation*}
    где $NA$ --- пропущенное значение. Также введем шаблон пропусков 
    \begin{equation*}
        Obs = (obs_1, obs_2, \dots, obs_T), obs_i \in \{ 0,1 \},
    \end{equation*}
    где $obs_i = 0$ значит, что $x_i = NA$, а $obs_i = 1$ значит, что $x_i \ne NA$, и заранее известную функцию
    вероятности пропуска элемента со значением $x$
    \begin{equation*}
        m(x) = P\{obs = 0 \mid \xi = x\}, \quad x \in \Rb .
    \end{equation*}
    Таким образом, получим $\displaystyle P\{obs_i = 1\} = \int_{-\infty}^{+\infty} p(x)(1 - m(x))\dd x$.

    В рамках вышеописанной модели \textit{случайными пропусками} будем называть случай $m(x) = \text{const}$. 
    \textit{Полным цензурированием справа} будем называть случай
    \begin{equation*}
        m(x) = \left\{
        \begin{array}{ll}
            0, & x \leq c, \\
            1, & x < c
        \end{array}\right. , \quad c \in \Rb .
    \end{equation*}
    Аналогично определим \textit{полное цензурирование слева}. Кроме данных выше моделей будет рассматриваться модель 
    с кусочно-постоянной функцией $m(x)$.

    В рамках курсовой работы была поставлена задача: по имеющимся наблюдениям оценить параметры распределения выборки 
    и показать, что стандартные методы, использующиеся для оценки параметров распределения по выборке без пропусков, 
    не дают результатов в случае пропусков, зависящих от данных. Для проверки работы различных методов проводились 
    компьютерные эксперименты.