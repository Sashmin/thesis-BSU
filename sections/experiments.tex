\section{Компьютерные эксперименты}

    Все компьютерные эксперименты проводятся по единому шаблону. Рассмотрим ход эксперимента.

    Создается случайная выборка размера $T$ с некоторым распределением $p_\xi$, зависящим от параметров. Затем 
    моделируется некоторая ситуация пропусков данных, описанная функцией $m(x)$.

    После происходит оценка параметров. В этом этапе содержатся основные отличия методов, испытываемых в курсовой 
    работе, поэтому их будем рассматривать индивидуально.

    Вышеописанная процедура проводится $n = 500$ раз для размеров 
    \[
        T = \overline{50, 100, \dotsc, 450, 500}.
    \]

    При этом для каждого $T$ считается выборочное среднее оценки параметра 
    \begin{equation*}
        E\{\theta\} = \frac{1}{500}\sum_{k=1}^{500}\hat{\theta}_k
    \end{equation*}
    и выборочное среднее вариации оценки
    \begin{equation*}
        E\{\mathrm{var}(\theta)\} = \frac{1}{500}\sum_{k=1}^{500}(\hat{\theta}_k - \theta)^2.
    \end{equation*}