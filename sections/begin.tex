\section{Введение}

    Временной ряд представляет собой набор наблюдений, полученных путем регулярного измерения одной переменной
    в течение некоторого периода времени. Наблюдения представляют собой набор из одного или нескольких значений, 
    зафиксированных в определенный момент времени. Элементами наблюдения являются действительные числа --- значения 
    непрерывных или дискретных переменных.

	Временные ряды --- один из наиболее важных инструментов в аналитике данных. Они используются во многих областях, 
    включая финансы, производство, социальные и экономические исследования, климатологию и другие. Примеры временных 
    рядов могут включать данные о продажах продукции в определенный день или данные о температуре на определенной 
    территории в различные временные промежутки.

	На практике зачастую часть значений переменных во временном ряду по какой-то причине отсутствует. Например, 
    часть респондентов, участвующих в обследовании семей, может отказаться сообщить размер дохода. Пропуски также 
    могут быть вызваны факторами, не связанными с самим экспериментальным процессом, например, ограничения или 
    неисправности оборудования, собирающего данные.
    
    Существуют множество видов пропусков, но в литературе наиболее часто встречаются следующие:
    \begin{itemize}
        \item Случайные пропуски --- пропуски, не зависящие от данных и от самого эксперимента (второй пример 
        из приведенных выше)
        \item Цензурирование --- пропуски, зависящие от данных, при которых пропускаются лишь значения из определенной 
        области. Цензурирование называется полным, если пропускаются все значения из области (в качестве примера 
        можно взять исследование сроков наступления эффекта лекарства, в котором все значения, большие срока 
        проведения эксперимента, будут пропущены). Цензурирование называется частичным если лишь часть значений из 
        области пропускается (пример с размером дохода в опросе семей).
    \end{itemize}

    При работе с пропусками, зависящими от данных, могут возникнуть серьезные проблемы, так как в этом случае 
    большинство рядовых методов, используемых для статистического анализа временных рядов, как игнорирование 
    пропусков, оказываются неэффективными, что будет подтверждено в ходе курсового проекта. Поэтому необходимо 
    использование иных методов, основывающихся на некоторой определенной модели пропусков или на более общей модели.
